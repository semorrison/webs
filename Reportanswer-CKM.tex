\documentclass{amsart}

\usepackage[dvips]{graphics,color}

\usepackage{amsthm}



\setlength{\topmargin}{0cm}

\setlength{\textwidth}{6.5in}

\setlength{\oddsidemargin}{0cm}

\setlength{\evensidemargin}{0cm}

\setlength{\textheight}{8.5in}

\setlength{\headsep}{0.5cm}



\newtheorem{thm}{Theorem}[section]

\newtheorem{lem}[thm]{Lemma}

\newtheorem{claim}[thm]{Claim}

\newtheorem{fact}[thm]{Fact}

\newtheorem{cor}[thm]{Corollary}

\newtheorem{prop}[thm]{Proposition}

\newtheorem{conj}[thm]{Conjecture}

\newtheorem{ex}[thm]{Example}



\def\C{{\mathbb C}}

\def\P{{\mathbb P}}

\def\R{{\mathbb R}}

\def\Z{{\mathbb Z}}

\def\O{{\mathcal O}}

\def\Fact{{\bf Fact:}}

\def\Claim{{\bf Claim:}}

\def\Example{{\bf Example:}}

\def\Note{{\bf Note: }}



\title{Responses to Referee's Report: Webs and quantum skew Howe duality}



\begin{document}



\maketitle



Thank you very much to the referees for their careful reading of this paper.



Here are the responses to the remarks in the referee report.



\begin{enumerate}

\item We left $ q = 1 $ in section 1.1 and removed the quantum integers from section 1.2.   Also we clarified that $ q $ is taken to be a formal variable throughout the paper.  We do not believe that the results hold when $ q $ is a root of unity. 
    
\item Fixed typo.

\item Clarified what happens to the equations under mirror reflection etc. Also, fixed the calculation in Lemma 2.2.1. Indeed $(k,l,r,s)$ should be $(2,k,k,1)$. The weird signs are a consequence of a very early version of the paper when certain relations had signs. The current calculation is fairly straight forward.

\item Fixed typo to say ``labelled'' everywhere.

\item Fixed typo.

\item Fixed typo.

\item Fixed typo.

\item Fixed typo.

\item Fixed typo.

\item Yes, there were minus signs missing which are now fixed.

\item Fixed argument.

\item Fixed typos.

\item Added proof of this identity.

\item Explanation added.

\item Fixed typo.

\item Fixed typo.

\item There was a problem with the whole calculation which is now fixed. The problem is that we didn't use the right map for the cup. 

\item Fixed typo.

\item Fixed typo.

\item Added the referee's suggested explanation in order to help reader.

\item Fixed typo.

\item Fixed typo.

\item Added a reference to the suggested paper regarding this result.
\item Fixed typo.

\item Improved the exposition, by proving (5.3) first and then using that to explain why the special case suffices.

\item Fixed typo.

\item Added the second hexagon equation.

\item Fixed typo.

\item Added reference.

\item Added subscripts.

\item Erased ``that''.

\item Fixed: ``thus be the hexagon equation\emph{s}''

\item NEED

\item Fixed.

\item NEED

\end{enumerate}



\end{document}

