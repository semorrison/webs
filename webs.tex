\documentclass[11pt,leqno]{article}

%auto-ignore
%!TEX root = webs.tex
%this ensures the arxiv doesn't try to start TeXing here.

\usepackage{amsmath,amssymb,amsfonts,amsthm}
\usepackage{ifpdf}

\usepackage{comment}

\usepackage[all]{xy}
\SelectTips{cm}{}
% This may speed up compilation of complex documents with many xymatrices.
%\CompileMatrices

\usepackage[section]{placeins}
\usepackage{leftidx}
\usepackage{stmaryrd} % additional math symbols, e.g. \mapsfrom
%\usepackage{libertine}
%\usepackage[T1]{fontenc}
\usepackage{microtype}

% ----------------------------------------------------------------
\vfuzz5pt % Don't report over-full v-boxes if over-edge is small
\hfuzz5pt % Don't report over-full h-boxes if over-edge is small
% ----------------------------------------------------------------

% don't warn about PDF 1.5 (default was 1.4); dangerous?
\pdfminorversion=5

% diagrams -------------------------------------------------------
% figures ---------------------------------------------------------
\newcommand{\pathtotrunk}{./}
\newcommand{\pathtodiagrams}{\pathtotrunk}

\newcommand{\mathfig}[2]{\ensuremath{\hspace{-3pt}\begin{array}{c}%
  \raisebox{-2.5pt}{\includegraphics[width=#1\textwidth]{\pathtodiagrams #2}}%
\end{array}\hspace{-3pt}}}
\newcommand{\reflectmathfig}[2]{{\hspace{-3pt}\begin{array}{c}%
  \raisebox{-2.5pt}{\reflectbox{\includegraphics[width=#1\textwidth]{\pathtodiagrams #2}}}%
\end{array}\hspace{-3pt}}}
\newcommand{\rotatemathfig}[3]{{\hspace{-3pt}\begin{array}{c}%
  \raisebox{-2.5pt}{\rotatebox{#2}{\includegraphics[height=#1\textwidth]{\pathtodiagrams #3}}}%
\end{array}\hspace{-3pt}}}
\newcommand{\placefig}[2]{\includegraphics[width=#1\linewidth]{\pathtodiagrams #2}}

\newcommand{\arxiv}[1]{\href{http://arxiv.org/abs/#1}{\tt arXiv:\nolinkurl{#1}}}
\newcommand{\doi}[1]{\href{http://dx.doi.org/#1}{{\tt DOI:#1}}}
\newcommand{\euclid}[1]{\href{http://projecteuclid.org/euclid.cmp/#1}{{\tt #1}}}
\newcommand{\mathscinet}[1]{\href{http://www.ams.org/mathscinet-getitem?mr=#1}{\tt #1}}
\newcommand{\googlebooks}[1]{(preview at \href{http://books.google.com/books?id=#1}{google books})}


% THEOREMS -------------------------------------------------------
\theoremstyle{plain}
%\newtheorem*{fact}{Fact}
\newtheorem{prop}{Proposition}[subsection]
\makeatletter
\@addtoreset{prop}{section}
\makeatother
\newtheorem{conj}[prop]{Conjecture}
\newtheorem{thm}[prop]{Theorem}
\newtheorem{lem}[prop]{Lemma}
\newtheorem*{lem*}{Lemma}
\newtheorem{cor}[prop]{Corollary}
\newtheorem*{cor*}{Corollary}
\newtheorem*{exc}{Exercise}
\newtheorem{defn}[prop]{Definition}         % numbered definition
\newtheorem*{defn*}{Definition}             % unnumbered definition
\newtheorem{question}{Question}
\newtheorem{property}[prop]{Property}
\newenvironment{rem}{\vspace{0.3cm}\noindent\textsl{Remark.}}{}  % perhaps looks better than rem above?
\newenvironment{example}{\vspace{0.3cm}\noindent\textbf{Example.}}{}  % perhaps looks better than rem above?
\newtheorem{rem*}[prop]{Remark}
\numberwithin{equation}{section}
%% example, claim and remark are defined in article_preamble.tex, for compatibility with beamer and PNAS


% Marginal notes in draft mode -----------------------------------
\newcounter{comment}
\newcommand{\noop}[1]{}
\newcommand{\todo}[1]{\textbf{\color[rgb]{.8,.2,.5}\small TODO: #1}}

% \mathrlap -- a horizontal \smash--------------------------------
% For comparison, the existing overlap macros:
% \def\llap#1{\hbox to 0pt{\hss#1}}
% \def\rlap#1{\hbox to 0pt{#1\hss}}
\def\clap#1{\hbox to 0pt{\hss#1\hss}}
\def\mathllap{\mathpalette\mathllapinternal}
\def\mathrlap{\mathpalette\mathrlapinternal}
\def\mathclap{\mathpalette\mathclapinternal}
\def\mathllapinternal#1#2{%
\llap{$\mathsurround=0pt#1{#2}$}}
\def\mathrlapinternal#1#2{%
\rlap{$\mathsurround=0pt#1{#2}$}}
\def\mathclapinternal#1#2{%
\clap{$\mathsurround=0pt#1{#2}$}}

% MATH -----------------------------------------------------------
\newcommand{\id}{\boldsymbol{1}}
\renewcommand{\imath}{\mathfrak{i}}
\renewcommand{\jmath}{\mathfrak{j}}

\newcommand{\ssum}[1]{\Sigma#1}
\newcommand{\sumhat}{\overline{\sum}}
\newcommand{\sumtah}{\underline{\sum}}

\newcommand{\lmod}[1]{\leftidx{_{#1}}{\operatorname{mod}}{}}

\newcommand{\into}{\hookrightarrow}
\newcommand{\onto}{\twoheadrightarrow}
\newcommand{\iso}{\cong}
\newcommand{\quism}{\underset{\text{q.i.}}{\simeq}}
\newcommand{\htpy}{\simeq}
\newcommand{\actsOn}{\circlearrowright}
\newcommand{\xto}[1]{\xrightarrow{#1}}
\newcommand{\isoto}{\xto{\iso}}
\newcommand{\quismto}{\xrightarrow[\text{q.i.}]{\iso}}
\newcommand{\diffeoto}{\xrightarrow[\text{diffeo}]{\iso}}
\newcommand{\htpyto}{\xrightarrow[\text{htpy}]{\htpy}}

\newcommand{\restrict}[2]{#1{}_{\mid #2}{}}
\newcommand{\set}[1]{\left\{#1\right\}}
\newcommand{\setc}[2]{\setcl{#1}{#2}}
\newcommand{\setcl}[2]{\left\{ \left. #1 \;\right| \; #2 \right\}}
\newcommand{\setcr}[2]{\left\{ #1 \;\left| \; #2 \right\}\right.}

\newcommand{\floor}[1]{\left\lfloor#1\right\rfloor}
\newcommand{\norm}[1]{\left|\left|#1\right|\right|}
\newcommand{\abs}[1]{\left|#1\right|}

\newcommand{\qi}[2][q]{\left[#2\right]_{#1}}
\newcommand{\qBinomial}[3][q]{\genfrac{[}{]}{0pt}{}{#2}{#3}_{#1}}
\newcommand{\qPoch}[3]{\left(#1;#2\right)_{#3}}

\newcommand{\card}[1]{\sharp{#1}}

\newcommand{\bdy}{\partial}
\newcommand{\compose}{\circ}
\newcommand{\eset}{\emptyset}

\newcommand{\directSum}{\oplus}
\newcommand{\DirectSum}{\bigoplus}
\newcommand{\tensor}{\otimes}
\newcommand{\Tensor}{\bigotimes}

\newcommand{\Homa}[3]{\Hom_{#1}\left(#2,#3\right)}
\newcommand{\Hom}{\operatorname{Hom}}
\newcommand{\End}[1]{\operatorname{End}\left(#1\right)}

% ----------------------------------------------------------------

\ifpdf
	\usepackage[pdftex,plainpages=false,hypertexnames=false,pdfpagelabels]{hyperref}
	\usepackage[pdftex]{graphicx}
\else
	\usepackage[plainpages=false,hypertexnames=false,pdfpagelabels]{hyperref}
	\usepackage{graphicx}
\fi

%must load tikz after graphicx
\usepackage{tikz}
\usetikzlibrary{shapes}
\usetikzlibrary{backgrounds}
\usetikzlibrary{decorations,decorations.pathreplacing,decorations.markings}
\usetikzlibrary{fit,calc,through}
\usetikzlibrary{external}

\tikzstyle{mid>}=[decoration={markings, mark=at position 0.5 with {\arrow{>}}}, postaction={decorate}]
\tikzstyle{mid<}=[decoration={markings, mark=at position 0.5 with {\arrow{<}}}, postaction={decorate}]
\tikzstyle{upper>}=[decoration={markings, mark=at position 0.8 with {\arrow{>}}}, postaction={decorate}]
\tikzstyle{upper<}=[decoration={markings, mark=at position 0.8 with {\arrow{<}}}, postaction={decorate}]
\tikzstyle{lower>}=[decoration={markings, mark=at position 0.2 with {\arrow{>}}}, postaction={decorate}]
\tikzstyle{lower<}=[decoration={markings, mark=at position 0.2 with {\arrow{<}}}, postaction={decorate}]

\def\Foam{{\mathcal{F}{\rm oam}}}
\newcommand{\alt}{\wedge}
\newcommand{\Alt}[2]{{\textstyle\bigwedge^{#1}_{#2}}}
\newcommand{\Usl}[1]{U\sl_{#1}}
\newcommand{\one}{1}
\def\sA{\mathcal{A}}
\def\l{\lambda}
\def\bZ{{\mathbb{Z}}}
\def\sl{{\mathfrak{sl}}}
\def\Sp{{\mathcal{S}p}}
\def\FSp{{\mathcal{FS}p}}
\def\bC{{\mathbb{C}}}
\def\g{{\mathfrak{g}}}
\def\SL{{\rm{SL}}}
\def\GL{{\rm{GL}}}
\def\gl{{\mathfrak{gl}}}
\def\dU{\dot{{\mathcal{U}}}_q}
\def\Uq{\mathcal{U}_q}
\def\Rep{\mathcal{R}ep}
\def\la{\langle}
\def\ra{\rangle}
\def\dalg{\dot{{U}}_q}

\newcommand{\ul}[1]{{\underline{#1}}}

%\newcommand{\RepSL}[1]{\mathcal{R}ep(SL_{#1})}
\newcommand{\Lad}{\mathcal{L}ad}

\usepackage{environ}
\usepackage{xargs}

\newcommandx{\NewEnvironx}[5][2,3]{%
  \expandafter\newcommandx\csname start#1\endcsname[#2][#3]{#4}%
  \NewEnviron{#1}{\csname start#1\expandafter\endcsname\BODY #5}}

\newcommand{\ladderX}{1.5}
\newcommand{\ladderY}{1.5}
\newcommand{\ladderR}{0.6}
\newcommand{\laddercoordinates}[2]{
\foreach \x in {0,...,#1} {
	\foreach \y in {0,...,#2} {
		\coordinate (l\x\y) at (\x * \ladderX, \y * \ladderY);
		\coordinate (u\x\y) at ($(l\x\y)+\ladderR*(0,\ladderY)$);
		\coordinate (d\x\y) at ($(l\x\y)+(0,\ladderY)-\ladderR*(0,\ladderY)$);
	}
}
}
\newcommand{\ladderEn}[5]{
\draw[mid>] (l#1#2) -- (d#1#2);
\draw[mid>] (d#1#2) -- ($(l#1#2)+(0,\ladderY)$) node[left] {#3};
\draw[mid>] ($(l#1#2)+(\ladderX,0)$) -- ($(u#1#2)+(\ladderX,0)$);
\draw[mid>] ($(u#1#2)+(\ladderX,0)$) -- ($(l#1#2)+(\ladderX,\ladderY)$) node[right] {#4};
\draw[mid>] (d#1#2) --node[above]{#5} ($(u#1#2)+(\ladderX,0)$);
}
\newcommand{\ladderE}[4]{\ladderEn{#1}{#2}{#3}{#4}{}}
\newcommand{\ladderFn}[5]{
\draw[mid>] (l#1#2) -- (u#1#2);
\draw[mid>] (u#1#2) -- ($(l#1#2)+(0,\ladderY)$) node[left] {#3};
\draw[mid>] ($(l#1#2)+(\ladderX,0)$) -- ($(d#1#2)+(\ladderX,0)$);
\draw[mid>] ($(d#1#2)+(\ladderX,0)$) -- ($(l#1#2)+(\ladderX,\ladderY)$) node[right] {#4};
\draw[mid>] ($(d#1#2)+(\ladderX,0)$) --node[above]{#5} (u#1#2);
}
\newcommand{\ladderF}[4]{\ladderFn{#1}{#2}{#3}{#4}{}}
\newcommand{\ladderIn}[3]{\draw[mid>] (l#1#2) -- +($#3*(0,\ladderY)$);}
\newcommand{\ladderI}[2]{\ladderIn{#1}{#2}{1}}

\NewEnvironx{ladder}[2]{%
  \begin{tikzpicture}[baseline=13*\ladderY*#2]\laddercoordinates{#1}{#2}}
{\end{tikzpicture}}

\newcommand{\fuse}[3]{\tikz[baseline=0.5cm]{
\coordinate (z1) at (0,0);
\coordinate (z2) at (1,0);
\coordinate (c) at (0.5,0.5);
\coordinate (e) at (0.5,1);
\draw[mid>] (z1) node[below] {$#1$} -- (c);
\draw[mid>] (z2) node[below] {$#2$} -- (c);
\draw[mid>] (c) -- (e) node[above] {$#3$};
}}
\newcommand{\fork}[3]{\tikz[baseline=0.5cm]{
\coordinate (z1) at (0,1);
\coordinate (z2) at (1,1);
\coordinate (c) at (0.5,0.5);
\coordinate (e) at (0.5,0);
\draw[mid<] (z1) node[above] {$#1$} -- (c);
\draw[mid<] (z2) node[above] {$#2$} -- (c);
\draw[mid<] (c) -- (e) node[below] {$#3$};
}}


% example for creating tikz environments compatible with externalize
% thanks Andrew Stacey: http://tex.stackexchange.com/a/15614/77
%\NewEnvironx{mytikz}[1][1=]{%
%  \begin{figure}[htp]
%  \centering
%  \begin{tikzpicture}[#1]}
%{\end{tikzpicture}
%  \end{figure}}

% tricky way to iterate macros over a list
\def\semicolon{;}
\def\applytolist#1{
    \expandafter\def\csname multi#1\endcsname##1{
        \def\multiack{##1}\ifx\multiack\semicolon
            \def\next{\relax}
        \else
            \csname #1\endcsname{##1}
            \def\next{\csname multi#1\endcsname}
        \fi
        \next}
    \csname multi#1\endcsname}

% \def\cA{{\cal A}} for A..Z
\def\calc#1{\expandafter\def\csname c#1\endcsname{{\mathcal #1}}}
\applytolist{calc}QWERTYUIOPLKJHGFDSAZXCVBNM;

\usepackage{color}

% idea from tex-overflow
\usepackage{xcolor}
\definecolor{dark-red}{rgb}{0.7,0.25,0.25}
\definecolor{dark-blue}{rgb}{0.15,0.15,0.55}
\definecolor{medium-blue}{rgb}{0,0,0.65}
\hypersetup{
    colorlinks, linkcolor={dark-red},
    citecolor={dark-blue}, urlcolor={medium-blue}
}


% margin stuff
\setlength{\textwidth}{6.5in}
\setlength{\oddsidemargin}{0in}
\setlength{\evensidemargin}{0in}
\setlength{\textheight}{8.5in}
\setlength{\topmargin}{-.25in}



\title{something about webs and skew Howe duality}
\author{Sabin~Cautis, Joel Kamnitzer and Scott~Morrison}

\begin{document}

\makeatletter
\@addtoreset{equation}{section}
\gdef\theequation{\thesection.\arabic{equation}}
\makeatother

\maketitle

\begin{abstract}
\end{abstract}

\hypersetup{
    colorlinks, linkcolor={black},
    citecolor={dark-blue}, urlcolor={medium-blue}
}

\tableofcontents

\hypersetup{
    colorlinks, linkcolor={dark-red},
    citecolor={dark-blue}, urlcolor={medium-blue}
}

\newcommand{\Alt}{\bigwedge}

\renewcommand{\sl}[1]{\mathfrak{sl}_{#1}}
\newcommand{\Usl}[1]{U\sl{#1}}

\section{Introduction}
The representation theory of $\sl{n}$ is a pivotal tensor category, and it is natural to ask for a presentation by generators and relations, as a pivotal tensor category.

There are two main choices one needs to make before looking for such a presentation. First, it would be reasonable to pass to any full subcategory, whose idempotent completion recovers the entire representation theory. In particular, in this paper we look at the full subcategory whose objects are tensor products of the fundamental representations $\Alt^k \mathbb C^n$ of $\sl{n}$. (It might alternatively be interesting to descend all the way to the full subcategory whose objects are tensor powers of the standard representation.) Second, we need to decide which generators to use. We take the maps $\Alt^a \mathbb{C}^n \tensor \Alt^b \mathbb{C}^n \tensor \Alt^c \mathbb{C}^n \to \mathbb{C}$. (The space of such maps is one-dimensional if $a+b+c$ is a multiple of $n$, or zero-dimensional otherwise.) It is relatively easy to show that these are indeed generators, i.e. that every $\sl{n}$-linear map between tensor products of fundamental representations can be written as tensor products and compositions of these maps, along with the duality pairing and copairing maps. The question then, is to identify the relations holding between such compositions.

Said another way, we have a pivotal category of trivalent webs, with oriented edges labelled by $\{0, \ldots, n\}$, and at each vertex the labels summing to a multiple of $n$, and a functor to the representation theory category. The question is to identify the pivotal ideal which is the kernel of this functor.

This problem has been studied previously. For $n=2$, there are no trivalent vertices, and the category of webs is essentially just the category of embedded 1-manifolds up to isotopy. The kernel of the functor to representation theory is the ideal generated by the difference $\tikz[baseline=-2pt]{\node[draw,circle] {};} - 2$. \todo{Explain that this is Temperley-Lieb.}

For $n=3$ .... \todo{talk about Kuperberg's relations for $\sl{3}$.}

For $n \geq 4$, generators for the kernel have been proposed, by \cite{} (for $n=4$) and by \cite{} (generally), but without proving that their lists of relations were complete.

This paper answers the question, in particular showing that the relations of \cite{} are complete. (In fact, unnecessarily complete; just the $I=H$ and `square-switch' relations suffice, and generate the others.) To do this, we make use of skew Howe duality. In fact we give a very succinct recipe for the relations, as certain truncations of relations holding in $\Usl{m}$. We now give a quick overview of the argument.

We denote the quotient of the category of trivalent webs by the $I=H$ and square switch relations as $\cW$. We have a surjective functor to the the representation theory, which we would like to show is an isomorphism. All that remains is to check that it is injective on morphisms.

Skew Howe duality states that the commuting actions of $\Usl{n}$ and $\Usl{m}$ on $\Alt^\bullet(\mathbb{C}^n \tensor \mathbb{C}^m)$ are in fact each the commutant of the other. That is, if  $f: \Alt^\bullet(\mathbb{C}^n \tensor \mathbb{C}^m) \to \Alt^\bullet(\mathbb{C}^n \tensor \mathbb{C}^m)$ is $\sl{n}$-linear, then there is some element $X_f \in \Usl{m}$ whose action on  $\Alt^\bullet(\mathbb{C}^n \tensor \mathbb{C}^m)$ is exactly $f$.

Suppose we have some element $A \in \Hom{\cW}{\underline{k}}{\underline{k'}}$ which maps to zero in the representation theory. (Here $\underline{k}$ and $\underline{k}$ denote two sequences of integers in $\{0,\ldots,n\}$, i.e. two objects in $\cW$.) We would like to show that $A$ is itself zero. We first define a related element $\tilde{A} \in \Hom{\cW}{\underline{l}}{\underline{l'}}$, where $\underline{l}$ is obtained from $\underline{k}$ by interposing some number of $0$s and $n$s, and similarly for $\underline{l'}$, and further $\underline{l}$ and $\underline{l'}$ have the same length $m$ and the same sum. (Previously, the sums of $\underline{k}$ and $\underline{k'}$ might be differed by a multiple of $n$.)


\section{Skew Howe Duality}

\section{The kernel of $\Usl{m}$ acting on $\wedge^\bullet(\mathbb C^{n} \times \mathbb C^m)$}

\section{Web relations}
%In this section, we analyze each algebra relation in $\Usl{m}$, 


% ----------------------------------------------------------------
%\newcommand{\urlprefix}{}
\bibliographystyle{alpha}
\bibliography{bibliography/bibliography}
% ----------------------------------------------------------------

% ----------------------------------------------------------------
\end{document}
% ----------------------------------------------------------------

